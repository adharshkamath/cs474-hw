\documentclass[12pt,letterpaper, onecolumn]{exam}
\usepackage{amsmath}
\usepackage{amssymb}
\usepackage{listings}
\usepackage{pythonhighlight}
\usepackage[a4paper,lmargin=30pt, rmargin=50pt, tmargin=0.65in]{geometry}  %For centering solution box
% \chead{\hline} % Un-comment to draw line below header
\thispagestyle{empty}   %For removing header/footer from page 1

\begin{document}

\begingroup
\centering
\LARGE CS 474\\
\large Assignment 2 \\[0.5em]
\endgroup
\begingroup
\normalsize \quad\quad\quad Name: Adharsh Kamath \quad\quad\quad  UIN: 671259918 \quad\quad\quad  NetID: ak128 \par\
\endgroup
\rule{17cm}{0.4pt}
\pointsdroppedatright   %Self-explanatory
\printanswers
\renewcommand{\solutiontitle}{\noindent\textbf{Soln:}\enspace}
\newcommand{\cheading}[1]{{\underline{\textit{#1}}}}

\renewcommand{\questionshook}{%
	\setlength{\leftmargin}{18pt}%
	\setlength{\labelwidth}{-\labelsep}%
}
\qformat{\underline{Problem \thequestion}}
\begin{questions}
	\question[]
	\solutiontitle

	Let us use the symbol $\psi$ to refer to the given formula.
	\begin{align*}
		\psi = \left( p \land  (p \Rightarrow q) \right) \Rightarrow q
	\end{align*}

	In order to show that $\psi$ is valid, we can show that 
	\begin{align*}
		\neg \psi = \neg \left( (p \land  (p \Rightarrow q)  \Rightarrow q) \right)
	\end{align*}
	is unsatisfiable. Rewriting the above formula:
	\begin{align*}
		\neg \psi = \neg \left( (p \land  (p \Rightarrow q)  \Rightarrow q) \right) \\
					\equiv \neg \left( (p \land  (\neg p \lor q) ) \Rightarrow q \right) \\
					\equiv \neg \left( \neg (p \land  (\neg p \lor q) ) \lor q \right) \\
					\equiv (p \land  (\neg p \lor q) ) \land \neg q
	\end{align*}

	The last step is due to De Morgan's Law. We can now convert this to CNF, and construct a resolution refutation to show that it is unsatisfiable.
	To convert to CNF, we use the Tseitin transformation. We only need three new propositional variables, $x_{\psi}, x_1, x_2$,
	where $x_{\psi}$ corresponds to $\psi$, $x_1$ corresponds to $(\neg p \lor q)$ and $x_2$ corresponds to $(p \land x_1)$.
	This gives us the following set of clauses:
	\begin{align*}
		\left .
			\begin{cases}
					\{ \neg x_{\psi} \}, \\
					\{ \neg \neg x_{\psi}, x_2 \}, \{ \neg \neg x_{\psi}, \neg q \}, \{ \neg x_{\psi}, \neg x_2, \neg \neg q \} \\
					\{ \neg x_2, p \}, \{ \neg x_2, x_1 \}, \{ x_2, \neg p, \neg x_1  \}, \\
					\{ x_1, \neg \neg p \}, \{ x_1, \neg q \}, \{ \neg x_1, \neg p, q \}, 
			\end{cases}
		\right\}
	\end{align*}

	Simplifying the set by replacing $ \neg \neg p $ with $ p $ for all propositional variables, we get:
	\begin{align*}
		\left .
			\begin{cases}
				\{ \neg x_{\psi} \}, \\
				\{ x_{\psi}, x_2 \}, \{ x_{\psi}, \neg q \}, \{ \neg x_{\psi}, \neg x_2, q \} \\
				\{ \neg x_2, p \}, \{ \neg x_2, x_1 \}, \{ x_2, \neg p, \neg x_1  \}, \\
				\{ x_1, p \}, \{ x_1, \neg q \}, \{ \neg x_1, \neg p, q \},
			\end{cases}
		\right\}
	\end{align*}

	We can now create a resolution refutation to show that this set is unsatisfiable:
	\begin{align*}
		& 1. \{ \neg x_{\psi} \} \\
		& 2. \{ x_{\psi}, x_2 \} \\
		& 3. \{ x_2 \} \quad \text{Resolvent of 1 and 2} \\
		& 4. \{ \neg x_2, p \} \\
		& 5. \{ p \} \quad \text{Resolvent of 3 and 4} \\
		& 6. \{ \neg x_1, \neg p, q \} \\
		& 7. \{ \neg x_1, q \} \quad \text{Resolvent of 5 and 6} \\
		& 8. \{ x_{\psi}, \neg q \} \\ 
		& 9. \{ x_{\psi}, \neg x_1 \} \quad \text{Resolvent of 7 and 8} \\
		& 10. \{ \neg x_1 \} \quad \text{Resolvent of 1 and 9} \\
		& 11. \{ \neg x_2, x_1 \} \\
		& 12. \{ \neg x_2 \} \quad \text{Resolvent of 3 and 11} \\
		& 13. \{  \} \quad \text{Resolvent of 10 and 12}
	\end{align*}

	By creating this resolution refutation, we have shown that there is no valuation that can satisfy $ \neg \psi $.
	Since $ \neg \psi $ is unsatisfiable, we can conclude that $ \psi $ is valid.

	TODO: Running resolution tool

    {\rule{17cm}{0.4pt}}
	\question[]
	\solutiontitle

	We are given the formula:
	\begin{align*}
		\psi = (q \lor \neg r) \land (\neg p \lor r) \land (\neg q \lor r \lor p) \land (p \lor q \lor \neg q) \land (\neg r \lor q)
	\end{align*}

	We can find resolvents in the following way:
	(Only resolutions that lead to clauses that do not already exist in the set of clauses are considered. 
	Resolutions that result in "trivial" clauses are also considered.)
	\begin{align*}
		& 1. \{ q \lor \neg r \} \\
		& 2. \{ \neg p \lor r \} \\
		& 3. \{ q \lor \neg p \} \quad \text{Resolvent of 1 and 2} \\
		& 4. \{ \neg q \lor r \lor p \} \\
		& 5. \{ r \lor \neg r \lor p \} \quad \text{Resolvent of 1 and 4} \\
		& 6. \{ p \lor q \lor \neg q \} \\
		& 7. \{ p \lor q \lor \neg r \} \quad \text{Resolvent of 1 and 6} \\
		& 8. \{ r \lor \neg q \} \quad \text{Resolvent of 2 and 4} \\
		& 9. \{ r \lor \neg r \} \quad \text{Resolvent of 2 and 5} \\
		& 10. \{ r \lor p \lor \neg p \} \quad \text{Resolvent of 2 and 5} \\
		& 11. \{ r \lor q \lor \neg q \} \quad \text{Resolvent of 2 and 6} \\
		& 12. \{ r \lor q \lor \neg r \} \quad \text{Resolvent of 2 and 7} \\
		& 12. \{ p \lor q \lor \neg p \} \quad \text{Resolvent of 2 and 7} \\
		& 13. \{  \}
	\end{align*}

    {\rule{17cm}{0.4pt}}
	\question[]
	\solutiontitle

    {\rule{17cm}{0.4pt}}

\end{questions}
\end{document}